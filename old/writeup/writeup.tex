\documentclass{article}
\usepackage[utf8]{inputenc}
\usepackage{amsmath,amsthm,amssymb,scrextend}
\usepackage[letterpaper, portrait, margin=1in]{geometry}
\usepackage{hyperref}
\usepackage{graphicx}
\usepackage{enumerate}
\usepackage{fancyhdr}
\usepackage{float}
\pagestyle{fancy}
\rhead{Daniel V. Rostovtsev : pg \thepage }
\usepackage{mathtools}
\DeclarePairedDelimiter\ceil{\lceil}{\rceil}
\DeclarePairedDelimiter\floor{\lfloor}{\rfloor}

\begin{document}

\section*{Problem Set 1}
Daniel V. Rostovtsev
\\Date: 17 April, 2018



\subsection*{Problem 1}
(SEE ATTACHED .txt FOR SOLUTION TO PROBLEM ONE, IT WAS DONE COMPUTATIONALLY)



\subsection*{Problem 2}
\begin{enumerate}[(a)]
  \item
    Show the matrix $A$ with an additional parity column. Explain how to
    recover the erased bits in $\hat{A}$ using the parity column. Can you
    recover them all?

    \vspace{5mm}
    Finging $A$ with parity column:
    \begin{tabular}{| c | c | c | c | c |}
      \hline
      0 & 1 & 1 & 1 & 1 \\
      \hline
      0 & 1 & 1 & 0 & 0 \\
      \hline
      1 & 1 & 1 & 1 & 0 \\
      \hline
      1 & 0 & 1 & 1 & 1 \\
      \hline
    \end{tabular}
    Note $\hat{A}$ is as follows:
    \begin{tabular}{| c | c | c | c | c |}
      \hline
      0 & 1 & 1 & E \\
      \hline
      0 & E & 1 & E \\
      \hline
      1 & 1 & 1 & E \\
      \hline
      1 & E & 1 & 1 \\
      \hline
    \end{tabular}

    Note that $\oplus$, or $XOR$ in binary can be considered formally as
    addition in the field $\mathbb{F}_2$. The only way to recover an erased
    bit is to find a unique solution to the equation
    \begin{align*}
      a_i + b_i + c_i + d_i = p_i \text{ (in $\mathbb{F}_2$)}
    \end{align*}
    where each $p_i$ is the parity bit, and $a_i,b_i,c_i,d_i$ are either the
    read bits or unknown erased bits. Since one linear equation can only yield
    a unique solution if there is only one unknown in a field (in this case,
    $\mathbb{F}_2$), it follows that there can only be one erasure per row if
    all the erased bits are to be recovered. So, using the parity bit, this is
    all that can be recovered from $\hat{A}$:

    \vspace{5mm}

    \begin{tabular}{| c | c | c | c | c |}
      \hline
      0 & 1 & 1 & (1) \\
      \hline
      0 & (1,0) & 1 & (0,1) \\
      \hline
      1 & 1 & 1 & (1) \\
      \hline
      1 & (0) & 1 & 1 \\
      \hline
    \end{tabular}
    (where $b_4$ is 1 if $b_2$ is 0, and $b_4$ is 0 if $b_2$ is 1)
  \item
    Assume that you have an $n \times n$ matrix with an additional parity
    column. We define an $erasure$ $pattern$ as a subset of entries in the
    $n \times (n+1)$ matrix with an additional parity column. What are the
    erasure patterns that can be corrected with an additional parity column?

    \vspace{5mm}

    Using the same argument in $A$ with linear equations in a field, but
    altered slightly to account for potential erasures of the parity bit.
    \begin{align*}
      a_i + b_i + c_i + d_i = p_i \implies a_i + b_i + c_i + d_i + p_i = 0
      \text{ in $\mathbb{F}_2$}
    \end{align*}
    Which is a linear equation in $\mathbb{F}_2$ including the parity bit.
    Thus a maximum of one bit can be erased per row, and all recoverable
    erasure patterns are patterns with no more than one bit erased per row.
  \item
    To improve your erasure correcting scheme you decide to add both a parity
    column and a parity row. Show the matrix $A$ with the additional parity
    column and parity row. Explain how to recover the erased bits in
    $\hat{A}$. Can you recover all of them?

    \vspace{5mm}

    $A$ with additional parity column and row:
    \begin{tabular}{| c | c | c | c | c |}
      \hline
      0 & 1 & 1 & 1 & 1 \\
      \hline
      0 & 1 & 1 & 0 & 0 \\
      \hline
      1 & 1 & 1 & 1 & 0 \\
      \hline
      1 & 0 & 1 & 1 & 1 \\
      \hline
      0 & 1 & 0 & 1 & \\
      \hline
    \end{tabular}

    \vspace{5mm}

    Each row or column with at least one unkown contributes to a system of
    linear equations in $\mathbb{F}_2$ with the computed parity bits. If the
    number such of linear equations is greater than or equal to the number
    of unknowns, then all the unknowns can be recovered. In the case of
    $\hat{A}$, there are 4 rows with at least one unknown, and 2 columns with
    at least one unknown. Thus there are 6 linear equations. There are 5
    erasures, so there are 5 unknowns. 6 linear equations can solve for 5
    unknowns in a field, so all $E$ can be recovered. This can be done with
    linear algebra, or just solving the grid like a sudoku puzzle. For
    $\hat{A}$ this is trivial. The first, third and fourth rows can be solved
    immediately with the parity column as in part (a). Then there is only one
    unknown in the second and fourth columns, which can be solved immediately
    with the additional parity row.
  \item
    Assume that the matrix $A$ with the additional parity column and parity
    row. We define an $erasure$ $pattern$ as a subset of entries in the
    $(n+1)\times(n+1)$ matrix. What are the erasure patterns that can be
    corrected in an $n \times n$ matrix with additional parity column and
    parity row.

    \vspace{5mm}

    Using the same logic as part (c), an erasure pattern will yield a system
    of equations with $r+c$ equations, where $r$ is the number of rows with
    an unknown, and $c$ is the number of columns. If the number of erasures,
    $n$ is less than or equal to $r+c$, then the erasure can be corrected.
  \item
    Assume that you stored the matrix $A$ with an additional parity column and
    parity row. Can you correct the error in $\tilde{A}$? What are the
    correctable error patterns that can be corrected in an $n \times n$ matrix
    with additional parity column and parity row?

    \vspace{5mm}

    No errors can be corrected with an additional parity column and parity row
    unless the matrix is a $1 \times 1$.

    \vspace{5mm}

    A matrix is correctable if the given parity column and rows are exactly
    able to identify the locations of all the errors, so that they can be
    altered. That means that given parity column and row errors correspond to
    unique errors. But, for $n \geq 2$, there are many $n \times n$ matrices
    that correspond to the same error code! We can show this by solving
    explicitly for the number of $n \times n$ matrices with a given error
    code. There are $n^2$ unknown entries of this matrix. Each entry in the
    parity column and parity row corresponds to a single linear equation on
    values of those $n^2$ unknown entries. There are only $2n$ such relations,
    $n$ coming from the parity row, and $n$ coming from the parity column. It
    follows that for $n \geq 3$, there are more unknowns than relations, so
    multiple matrices correspond to the same error, and can never be
    corrected. Now to check the special cases of $n = 2$ and $n = 1$.

    \vspace{5mm}

    For $n = 2$:
    \begin{tabular}{| c | c | c |}
      \hline
      $x_1$ & $x_2$ & $p_1$ \\
      \hline
      $x_3$ & $x_4$ & $p_2$ \\
      \hline
      $p_3$ & $p_4$ &   \\
      \hline
    \end{tabular}
    has a unique code if and only if the following equation is uniquely
    solvable:
    \begin{align*}
      \begin{pmatrix}
        1 & 1 & 0 & 0 \\
        0 & 0 & 1 & 1 \\
        1 & 0 & 1 & 0 \\
        0 & 1 & 0 & 1
      \end{pmatrix}
      \begin{pmatrix}
        x_1 \\ x_2 \\ x_3 \\ x_4
      \end{pmatrix} =
      \begin{pmatrix}
        p_1 \\ p_2 \\ p_3 \\ p_4
      \end{pmatrix}
    \end{align*}
    Which is true if and only if
    $A = \begin{pmatrix}
      1 & 1 & 0 & 0 \\
      0 & 0 & 1 & 1 \\
      1 & 0 & 1 & 0 \\
      0 & 1 & 0 & 1
    \end{pmatrix}$ is invertible in $M_{4 \times 4}(\mathbb{F}_2)$, which is
    true if and only if $\det(A)$ is nonzero in $\mathbb{F}_2$. But:
    \begin{align*}
      \det
      \begin{pmatrix}
        1 & 1 & 0 & 0 \\
        0 & 0 & 1 & 1 \\
        1 & 0 & 1 & 0 \\
        0 & 1 & 0 & 1
      \end{pmatrix}
      & =
          \det
          \begin{pmatrix}
            0 & 1 & 1 \\
            0 & 1 & 0 \\
            1 & 0 & 1
          \end{pmatrix}
          +
          \det
          \begin{pmatrix}
            1 & 0 & 0 \\
            0 & 1 & 1 \\
            1 & 0 & 1
          \end{pmatrix}
      & =
      \det
        \begin{pmatrix}
          1 & 1 \\
          1 & 0
        \end{pmatrix}
        +
        \det
        \begin{pmatrix}
          1 & 1 \\
          0 & 1
        \end{pmatrix}
      & = 1 + 1 = 0
    \end{align*}
    So no matrix of size $2 \times 2$ can be corrected. Obviously a matrix
    of size one can be corrected though. The entry is simply equal to the
    parity bit.
  \item
    (SEE ATTACHED .txt FOR SOLUTION TO (F)(i-iii), IT WAS DONE
    COMPUTATIONALLY)
  \item
    \begin{enumerate}[(i)]
      \item
        Is $[101,111]$ a single insertion correcting code? Prove your claim.
        Is $[000,101]$ a single insertion correcting code? Prove your claim.
        \begin{proof} Using pt (ii) of this section, and the preceding part,
          $[101,111]$ is not a single insertion correcting code since it is
          not single deletion correcting, and $[000,101]$ is single
          insertion correcting because it is single deletion correcting.
        \end{proof}
      \item
        Prove that a code is single insertion correcting if and only it is
        single deletion correcting.
        \begin{proof} Let $\{c_i\}$ be a code of length $l$. Let $D$ be the
          set of all deletions $d_k$ of the $k$-th entry if $c_i$, and let $I$
          be the set of all insertions $i_k(b)$ of a bit $b$ in the $k$-th
          entry of $c_i$. A code is single deletion correcting if $D(c_i)
          \cap D(c_j) = \emptyset$ for all $i \neq j$. Likewise, a code is
          single insertion correcting if $I(c_i) \cap I(c_j) = \emptyset$ for
          all $i \neq j$. Suppose a code is not single deletion correcting.
          Then there exists deletions $d_a, d_b$ such that $d_a(c_i) =
          d_b(c_j)$ for $i \neq j$. Let $i_a(b_1)$ and $i_b(b_2)$ be the
          reverse insertions of $d_a$ and $d_b$, and $i_c(b_3)$ be some
          arbitrary insertion. By definiton $i_a(b_1) \circ d_a (x)= i_b(b_2)
          \circ d_b (x)= x$, for all codewords $x$. It then follows that:
          \begin{align*}
            d_a(c_i) = d_b(c_j) & \implies i_c(b_3) \circ i_a(b_1) \circ
                d_a (c_i) = i_c(b_3) \circ i_b(b_2) \circ d_b (c_j) \\
              & \implies i_c(b_3)(c_i) = i_c(b_3)(c_j) \\
              & \implies I(c_i) \cap I(c_j) \neq \emptyset
          \end{align*}
          Which means that $\{c_i\}$ is not single insertion correcting,
          either. By contrapositive, this means single insertion correcting
          implies single deletion correcting. Now to go in the other
          direction. Suppose $\{c_i\}$ is not single insertion correcting.
          Then there exist $i_a(b_1), i_b(b_2) \in I$ such that $i_a(b_1)(c_i)
          = i_b(b_2)(c_j)$. Let $d_a, d_b \in D$ be the inverses of $i_a(b_1)$
          and $i_b(b_2)$, and let $d_c \in D$ be some arbitrary deletion. It
          then follows that:
          \begin{align*}
            i_a(b_1)(c_i) = i_b(b_2)(c_j) & \implies d_c \circ d_a \circ
                i_a(b_1)(c_i) = d_c \circ d_b \circ i_b(b_2)(c_j) \\
              & \implies d_c (c_i) = d_c (c_j) \\
              & \implies D(c_i) \cap D(c_j) \neq \emptyset
          \end{align*}
          Which means by contrapositive that single deletion correction
          implies single insertion correction. Now both directions have been
          proven, and it follows a code is single deletion correcting if and
          only if it is single insertion correcting.
        \end{proof}
    \end{enumerate}
\end{enumerate}



\end{document}
